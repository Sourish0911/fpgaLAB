\documentclass{article}
\usepackage[utf8]{inputenc}
\usepackage{circuitikz}
\usepackage{amsmath}
\usepackage{tabularx}
\usetikzlibrary{positioning}

\title{FPGA Assignment1}
\author{EE22MTECH02002}
\date{January 2022}

\begin{document}

\maketitle

\section{Question}
Differentiate Between Half adder and Full adder. Draw the Logic Diagram of full adder.

\section{Answer}

\begin{tabularx}{\textwidth} { | m{0.5cm} | m{5.3cm}| m{5cm} | }
 \hline
S. No. & HALF ADDER & FULL ADDER \\
 \hline
1. & The half adder produces a sum of the two inputs. & The full adder produces a sum of the two inputs and carry value. \\
\hline
2. &Previous carry is not used. & Previous carry is used. \\
 \hline
3. & It consists of one EX-OR gate and one AND gate. & It consists of two EX-OR, two AND gate and one OR gate. \\
 \hline
4. & Logical Expression for half adder is : S=A \oplus B;  C=A.B. & Logical Expression for Full adder is : S=A \oplus B \oplus Cin; Cout=(A.B)+(Cin.(A \oplus B)).\\
 \hline
 
\end{tabularx}


\vspace{15pt}
Full Adder Diagram.

\begin{figure}[ht]
   \centering 
   \begin{circuitikz}[scale=1]
   \draw (0,8) node[xor port](xor1){}
(-4.25, 8.25) node[anchor=east] {A} %%input A
(-4.25, 7.75) node[anchor=east] {B} %%input B
(-4, 6.75) node[anchor=east] {$\text{C}_{in}$} %%input C_in
(2, 7.5) node[xor port] (xor2){} 
(xor2.out) node[anchor=west] {S}
(2, 5.5) node[and port] (and1){}
(2, 3.5) node[and port] (and2){}
(4, 4.5) node[or port] (or){}
(-4, 8.25) to[short, -*](-3, 8.25)-|(xor1.in 1)
(-4, 7.75) to[short, -*](-2.5, 7.75)-|(xor1.in 2)
(-4, 6.75) to[short, -*](0.6, 6.75)-|(and1.in 1)
(0, 6.75)-|(xor2.in 2)
(0, 8) to[short, -*](0.25, 8)-|(xor2.in 1)

(0.25, 8)--(0.25, 5.25)-|(and1.in 2)
(-3, 8.25)--(-3, 3.25)-|(and2.in 2)
(-2.5, 7.75)--(-2.5, 3.75)-|(and2.in 1)
(and1.out)-|(or.in 1)
(and2.out)-|(or.in 2)
(or.out) node[anchor=west]{$\text{C}_{out}$}
;
\end{circuitikz}
\end{figure}

\begin{figure}[h!]
\resizebox{\columnwidth}{!}
    {
    

\tikzset{every picture/.style={line width=0.75pt}} %set default line width to 0.75pt        

\begin{tikzpicture}[x=0.75pt,y=0.75pt,yscale=-1,xscale=1]
%uncomment if require: \path (0,880); %set diagram left start at 0, and has height of 880

%Shape: Nand Gate [id:dp12621212173530783] 
\draw   (224.81,29) -- (247.04,29) .. controls (259.3,29) and (269.26,42.44) .. (269.26,59) .. controls (269.26,75.56) and (259.3,89) .. (247.04,89) -- (224.81,89) -- (224.81,29) -- cycle (210,39) -- (224.81,39) (210,79) -- (224.81,79) (278.15,59) -- (290,59) (269.26,59) .. controls (269.26,55.69) and (271.25,53) .. (273.7,53) .. controls (276.16,53) and (278.15,55.69) .. (278.15,59) .. controls (278.15,62.31) and (276.16,65) .. (273.7,65) .. controls (271.25,65) and (269.26,62.31) .. (269.26,59) -- cycle ;
%Shape: Nand Gate [id:dp37106405841343815] 
\draw   (224.81,166) -- (247.04,166) .. controls (259.3,166) and (269.26,179.44) .. (269.26,196) .. controls (269.26,212.56) and (259.3,226) .. (247.04,226) -- (224.81,226) -- (224.81,166) -- cycle (210,176) -- (224.81,176) (210,216) -- (224.81,216) (278.15,196) -- (290,196) (269.26,196) .. controls (269.26,192.69) and (271.25,190) .. (273.7,190) .. controls (276.16,190) and (278.15,192.69) .. (278.15,196) .. controls (278.15,199.31) and (276.16,202) .. (273.7,202) .. controls (271.25,202) and (269.26,199.31) .. (269.26,196) -- cycle ;
%Shape: Nand Gate [id:dp4400232901227137] 
\draw   (92.81,114) -- (115.04,114) .. controls (127.3,114) and (137.26,127.44) .. (137.26,144) .. controls (137.26,160.56) and (127.3,174) .. (115.04,174) -- (92.81,174) -- (92.81,114) -- cycle (78,124) -- (92.81,124) (78,164) -- (92.81,164) (146.15,144) -- (158,144) (137.26,144) .. controls (137.26,140.69) and (139.25,138) .. (141.7,138) .. controls (144.16,138) and (146.15,140.69) .. (146.15,144) .. controls (146.15,147.31) and (144.16,150) .. (141.7,150) .. controls (139.25,150) and (137.26,147.31) .. (137.26,144) -- cycle ;
%Shape: Nand Gate [id:dp18366865807960564] 
\draw   (374.81,101) -- (397.04,101) .. controls (409.3,101) and (419.26,114.44) .. (419.26,131) .. controls (419.26,147.56) and (409.3,161) .. (397.04,161) -- (374.81,161) -- (374.81,101) -- cycle (360,111) -- (374.81,111) (360,151) -- (374.81,151) (428.15,131) -- (440,131) (419.26,131) .. controls (419.26,127.69) and (421.25,125) .. (423.7,125) .. controls (426.16,125) and (428.15,127.69) .. (428.15,131) .. controls (428.15,134.31) and (426.16,137) .. (423.7,137) .. controls (421.25,137) and (419.26,134.31) .. (419.26,131) -- cycle ;
%Shape: Nand Gate [id:dp8531581332141267] 
\draw   (347.31,229.4) -- (369.54,229.4) .. controls (381.8,229.4) and (391.76,242.84) .. (391.76,259.4) .. controls (391.76,275.96) and (381.8,289.4) .. (369.54,289.4) -- (347.31,289.4) -- (347.31,229.4) -- cycle (332.5,239.4) -- (347.31,239.4) (332.5,279.4) -- (347.31,279.4) (400.65,259.4) -- (412.5,259.4) (391.76,259.4) .. controls (391.76,256.09) and (393.75,253.4) .. (396.2,253.4) .. controls (398.66,253.4) and (400.65,256.09) .. (400.65,259.4) .. controls (400.65,262.71) and (398.66,265.4) .. (396.2,265.4) .. controls (393.75,265.4) and (391.76,262.71) .. (391.76,259.4) -- cycle ;
%Straight Lines [id:da9769916942500323] 
\draw    (77.6,40.9) -- (78,124) ;
%Straight Lines [id:da010106615275015107] 
\draw    (77.6,40.9) -- (210,39) ;
%Straight Lines [id:da11423148537734718] 
\draw    (158,144) -- (209.6,143.9) ;
%Straight Lines [id:da3954444207951078] 
\draw    (210,79) -- (210,176) ;
%Straight Lines [id:da838471358783091] 
\draw    (78,164) -- (79.6,214.9) ;
%Straight Lines [id:da31995243036433973] 
\draw    (79.6,214.9) -- (162.6,214.9) -- (210,216) ;
%Straight Lines [id:da7779887429517995] 
\draw    (290,59) -- (359.6,58.9) ;
%Straight Lines [id:da7854810779746719] 
\draw    (359.6,58.9) -- (360,111) ;
%Straight Lines [id:da05249004576551064] 
\draw    (290,196) -- (359.4,197.9) ;
%Straight Lines [id:da05482357465177068] 
\draw    (360,151) -- (359.4,197.9) ;
%Straight Lines [id:da635562906139798] 
\draw    (176,144) -- (175,262.9) ;
%Straight Lines [id:da547000625013123] 
\draw    (175,262.9) -- (333,260.9) ;
%Straight Lines [id:da7205223743515847] 
\draw    (332.5,239.4) -- (332.5,279.4) ;
%Straight Lines [id:da6149942134607893] 
\draw    (51,124.9) -- (78,124) ;
%Straight Lines [id:da9171153629962425] 
\draw    (52,163.9) -- (78,164) ;
%Shape: Nand Gate [id:dp08441872317287213] 
\draw   (85.81,439) -- (108.04,439) .. controls (120.3,439) and (130.26,452.44) .. (130.26,469) .. controls (130.26,485.56) and (120.3,499) .. (108.04,499) -- (85.81,499) -- (85.81,439) -- cycle (71,449) -- (85.81,449) (71,489) -- (85.81,489) (139.15,469) -- (151,469) (130.26,469) .. controls (130.26,465.69) and (132.25,463) .. (134.7,463) .. controls (137.16,463) and (139.15,465.69) .. (139.15,469) .. controls (139.15,472.31) and (137.16,475) .. (134.7,475) .. controls (132.25,475) and (130.26,472.31) .. (130.26,469) -- cycle ;
%Shape: Nand Gate [id:dp7145888498990962] 
\draw   (198.81,371) -- (221.04,371) .. controls (233.3,371) and (243.26,384.44) .. (243.26,401) .. controls (243.26,417.56) and (233.3,431) .. (221.04,431) -- (198.81,431) -- (198.81,371) -- cycle (184,381) -- (198.81,381) (184,421) -- (198.81,421) (252.15,401) -- (264,401) (243.26,401) .. controls (243.26,397.69) and (245.25,395) .. (247.7,395) .. controls (250.16,395) and (252.15,397.69) .. (252.15,401) .. controls (252.15,404.31) and (250.16,407) .. (247.7,407) .. controls (245.25,407) and (243.26,404.31) .. (243.26,401) -- cycle ;
%Shape: Nand Gate [id:dp8577666747965609] 
\draw   (333.81,432) -- (356.04,432) .. controls (368.3,432) and (378.26,445.44) .. (378.26,462) .. controls (378.26,478.56) and (368.3,492) .. (356.04,492) -- (333.81,492) -- (333.81,432) -- cycle (319,442) -- (333.81,442) (319,482) -- (333.81,482) (387.15,462) -- (399,462) (378.26,462) .. controls (378.26,458.69) and (380.25,456) .. (382.7,456) .. controls (385.16,456) and (387.15,458.69) .. (387.15,462) .. controls (387.15,465.31) and (385.16,468) .. (382.7,468) .. controls (380.25,468) and (378.26,465.31) .. (378.26,462) -- cycle ;
%Shape: Nand Gate [id:dp33321173043143393] 
\draw   (221.81,508) -- (244.04,508) .. controls (256.3,508) and (266.26,521.44) .. (266.26,538) .. controls (266.26,554.56) and (256.3,568) .. (244.04,568) -- (221.81,568) -- (221.81,508) -- cycle (207,518) -- (221.81,518) (207,558) -- (221.81,558) (275.15,538) -- (287,538) (266.26,538) .. controls (266.26,534.69) and (268.25,532) .. (270.7,532) .. controls (273.16,532) and (275.15,534.69) .. (275.15,538) .. controls (275.15,541.31) and (273.16,544) .. (270.7,544) .. controls (268.25,544) and (266.26,541.31) .. (266.26,538) -- cycle ;
%Shape: Nand Gate [id:dp023537799853125874] 
\draw   (449.81,529) -- (472.04,529) .. controls (484.3,529) and (494.26,542.44) .. (494.26,559) .. controls (494.26,575.56) and (484.3,589) .. (472.04,589) -- (449.81,589) -- (449.81,529) -- cycle (435,539) -- (449.81,539) (435,579) -- (449.81,579) (503.15,559) -- (515,559) (494.26,559) .. controls (494.26,555.69) and (496.25,553) .. (498.7,553) .. controls (501.16,553) and (503.15,555.69) .. (503.15,559) .. controls (503.15,562.31) and (501.16,565) .. (498.7,565) .. controls (496.25,565) and (494.26,562.31) .. (494.26,559) -- cycle ;
%Straight Lines [id:da12954223044846835] 
\draw    (71.2,380.9) -- (71,449) ;
%Straight Lines [id:da21601794870041768] 
\draw    (71.2,380.9) -- (184,381) ;
%Straight Lines [id:da04657086548342204] 
\draw    (184,421) -- (184.2,518.9) ;
%Straight Lines [id:da9015616481473092] 
\draw    (184.2,518.9) -- (207,518) ;
%Straight Lines [id:da49062141479193544] 
\draw    (71,489) -- (71.2,556.9) ;
%Straight Lines [id:da47286256309412544] 
\draw    (71.2,556.9) -- (207,558) ;
%Straight Lines [id:da028279633074970167] 
\draw    (264,401) -- (319.2,400.9) ;
%Straight Lines [id:da3477244721002144] 
\draw    (319.2,400.9) -- (319,442) ;
%Straight Lines [id:da4990100642451052] 
\draw    (287,538) -- (319.2,538.9) ;
%Straight Lines [id:da7549165848982513] 
\draw    (319.2,538.9) -- (319,482) ;
%Shape: Nand Gate [id:dp5692789938616059] 
\draw   (528.81,451) -- (551.04,451) .. controls (563.3,451) and (573.26,464.44) .. (573.26,481) .. controls (573.26,497.56) and (563.3,511) .. (551.04,511) -- (528.81,511) -- (528.81,451) -- cycle (514,461) -- (528.81,461) (514,501) -- (528.81,501) (582.15,481) -- (594,481) (573.26,481) .. controls (573.26,477.69) and (575.25,475) .. (577.7,475) .. controls (580.16,475) and (582.15,477.69) .. (582.15,481) .. controls (582.15,484.31) and (580.16,487) .. (577.7,487) .. controls (575.25,487) and (573.26,484.31) .. (573.26,481) -- cycle ;
%Shape: Nand Gate [id:dp7323475602909904] 
\draw   (529.81,594) -- (552.04,594) .. controls (564.3,594) and (574.26,607.44) .. (574.26,624) .. controls (574.26,640.56) and (564.3,654) .. (552.04,654) -- (529.81,654) -- (529.81,594) -- cycle (515,604) -- (529.81,604) (515,644) -- (529.81,644) (583.15,624) -- (595,624) (574.26,624) .. controls (574.26,620.69) and (576.25,618) .. (578.7,618) .. controls (581.16,618) and (583.15,620.69) .. (583.15,624) .. controls (583.15,627.31) and (581.16,630) .. (578.7,630) .. controls (576.25,630) and (574.26,627.31) .. (574.26,624) -- cycle ;
%Shape: Nand Gate [id:dp9277421552146687] 
\draw   (609.01,518.7) -- (631.24,518.7) .. controls (643.5,518.7) and (653.46,532.14) .. (653.46,548.7) .. controls (653.46,565.26) and (643.5,578.7) .. (631.24,578.7) -- (609.01,578.7) -- (609.01,518.7) -- cycle (594.2,528.7) -- (609.01,528.7) (594.2,568.7) -- (609.01,568.7) (662.35,548.7) -- (674.2,548.7) (653.46,548.7) .. controls (653.46,545.39) and (655.45,542.7) .. (657.9,542.7) .. controls (660.36,542.7) and (662.35,545.39) .. (662.35,548.7) .. controls (662.35,552.01) and (660.36,554.7) .. (657.9,554.7) .. controls (655.45,554.7) and (653.46,552.01) .. (653.46,548.7) -- cycle ;
%Shape: Nand Gate [id:dp583615708679468] 
\draw   (566.81,671.8) -- (589.04,671.8) .. controls (601.3,671.8) and (611.26,685.24) .. (611.26,701.8) .. controls (611.26,718.36) and (601.3,731.8) .. (589.04,731.8) -- (566.81,731.8) -- (566.81,671.8) -- cycle (552,681.8) -- (566.81,681.8) (552,721.8) -- (566.81,721.8) (620.15,701.8) -- (632,701.8) (611.26,701.8) .. controls (611.26,698.49) and (613.25,695.8) .. (615.7,695.8) .. controls (618.16,695.8) and (620.15,698.49) .. (620.15,701.8) .. controls (620.15,705.11) and (618.16,707.8) .. (615.7,707.8) .. controls (613.25,707.8) and (611.26,705.11) .. (611.26,701.8) -- cycle ;
%Straight Lines [id:da8231151847817468] 
\draw    (399,462) -- (514,461) ;
%Straight Lines [id:da23328681193725442] 
\draw    (514,501) -- (515,604) ;
%Straight Lines [id:da5206884848610069] 
\draw    (594,481) -- (594.2,528.7) ;
%Straight Lines [id:da5457744859067841] 
\draw    (594.2,568.7) -- (595,624) ;
%Straight Lines [id:da6109593691746316] 
\draw    (508.15,559) -- (508.2,681.7) ;
%Straight Lines [id:da508815360172967] 
\draw    (508.2,681.7) -- (552,681.8) ;
%Straight Lines [id:da7198588081149899] 
\draw    (151,469) -- (184.1,469.95) ;
%Straight Lines [id:da6892147859658182] 
\draw    (36.6,581.5) -- (69,582.5) -- (435,579) ;
%Straight Lines [id:da4717401617067838] 
\draw    (398,578) -- (397,646.5) ;
%Straight Lines [id:da6587354665083593] 
\draw    (397,646.5) -- (515,644) ;
%Straight Lines [id:da09028325594788122] 
\draw    (397,646.5) -- (396,721.5) ;
%Straight Lines [id:da5210773934942985] 
\draw    (396,721.5) -- (552,721.8) ;
%Straight Lines [id:da6944442520766312] 
\draw    (436,462) -- (435,539) ;
%Straight Lines [id:da2472133994812351] 
\draw    (28.6,448.5) -- (71,449) ;
%Straight Lines [id:da5940082011897883] 
\draw    (29.6,489.5) -- (71,489) ;

% Text Node
\draw (33,116) node [anchor=north west][inner sep=0.75pt]   [align=left] {A};
% Text Node
\draw (33,158) node [anchor=north west][inner sep=0.75pt]   [align=left] {B};
% Text Node
\draw (447.2,124) node [anchor=north west][inner sep=0.75pt]   [align=left] {Sum};
% Text Node
\draw (420.2,251) node [anchor=north west][inner sep=0.75pt]   [align=left] {Carry};
% Text Node
\draw (35,9) node [anchor=north west][inner sep=0.75pt]   [align=left] {Half Adder NAND Implementation};
% Text Node
\draw (11,440) node [anchor=north west][inner sep=0.75pt]   [align=left] {A};
% Text Node
\draw (13,482) node [anchor=north west][inner sep=0.75pt]   [align=left] {B};
% Text Node
\draw (10,570) node [anchor=north west][inner sep=0.75pt]   [align=left] {Cin};
% Text Node
\draw (666.4,532) node [anchor=north west][inner sep=0.75pt]   [align=left] {Sum};
% Text Node
\draw (636.4,693) node [anchor=north west][inner sep=0.75pt]   [align=left] {Cout};
% Text Node
\draw (32,344) node [anchor=north west][inner sep=0.75pt]   [align=left] {Full Adder NAND Implementation};


\end{tikzpicture}

    }
\caption{Circuit Diagram of NAND Gate Implementation of Half Adder and Full Adder}
\end{figure}

\end{document}
